\batchmode
\makeatletter
\def\input@path{{/GD/dev/R/paperutils/inst/lyx//}}
\makeatother
\documentclass[11pt,british,super,sort&compress]{article}
\usepackage{mathptmx}
\usepackage{helvet}
\usepackage{courier}
\usepackage[T1]{fontenc}
\usepackage[latin9]{inputenc}
\setcounter{tocdepth}{2}
\synctex=-1
\usepackage{color}
\usepackage{babel}
\usepackage{array}
\usepackage{refstyle}
\usepackage{float}
\usepackage{url}
\usepackage{multirow}
\usepackage{graphicx}
\usepackage[unicode=true,pdfusetitle,
 bookmarks=true,bookmarksnumbered=false,bookmarksopen=false,
 breaklinks=true,pdfborder={0 0 0},backref=false,colorlinks=true]
 {hyperref}

\makeatletter

%%%%%%%%%%%%%%%%%%%%%%%%%%%%%% LyX specific LaTeX commands.

\AtBeginDocument{\providecommand\figref[1]{\ref{fig:#1}}}
\AtBeginDocument{\providecommand\tabref[1]{\ref{tab:#1}}}
%% Because html converters don't know tabularnewline
\providecommand{\tabularnewline}{\\}
\RS@ifundefined{subref}
  {\def\RSsubtxt{section~}\newref{sub}{name = \RSsubtxt}}
  {}
\RS@ifundefined{thmref}
  {\def\RSthmtxt{theorem~}\newref{thm}{name = \RSthmtxt}}
  {}
\RS@ifundefined{lemref}
  {\def\RSlemtxt{lemma~}\newref{lem}{name = \RSlemtxt}}
  {}


%%%%%%%%%%%%%%%%%%%%%%%%%%%%%% Textclass specific LaTeX commands.
\newcommand{\lyxaddress}[1]{
\par {\raggedright #1
\vspace{1.4em}
\noindent\par}
}

%%%%%%%%%%%%%%%%%%%%%%%%%%%%%% User specified LaTeX commands.
\usepackage[small]{caption}
%\usepackage{nonfloat}
\usepackage{pdflscape}
\setcounter{secnumdepth}{-1}
\newref{fig}{refcmd = {\hyperref[#1]{Fig.~\ref{#1}}}}
\newref{tab}{refcmd = {\hyperref[#1]{Table~\ref{#1}}}}
\usepackage{geometry}
\usepackage{pdfsync}
%\newrefformat{fig}{\hyperref[#1]{Fig.~\ref*{#1}}}

\makeatother

\begin{document}

\title{Some Title}


\author{Some Authors\textsuperscript{1}, Some Cheese\textsuperscript{1}}

\maketitle

\lyxaddress{\textsuperscript{1}Somewhere or other}
\begin{abstract}
A short summary

\newpage{}
\end{abstract}

\section{Introduction}

Some introductory thoughts


\section{Results}


\subsection{Results section 1}

Let's refer to \figref{wt}.


\subsection{Results section 2}

And here we'll refer to \tabref{S_cellcounts}.


\section{Discussion}


\section{Methods}


\subsection{Fly stocks}


\subsection{Analysis}


\section{Supplementary Information}

Supplementary Information is available for download at \url{http://somelab.org}.


\section{Acknowledgements}


\section{Author information}

The authors declare no competing financial interests. Correspondence
and requests for materials should be addressed to ABC (abc@xyz.edu).\newpage{}\newgeometry{text={183mm,247mm},centering}\pagestyle{empty}


\section{Figures}



\begin{figure*}[h]
\begin{centering}
\includegraphics{0_GD_dev_R_paperutils_inst_lyx_fig1.pdf}\caption{\label{fig:wt}\textbf{S}ome Figure}

\par\end{centering}

\textbf{\footnotesize a--b, }{\footnotesize some panels}
\end{figure*}


\newpage{}
\begin{figure*}[h]
\begin{centering}
\includegraphics{23_GD_dev_R_paperutils_inst_lyx_fig2.pdf}\caption{\label{fig:Or67d}Another figure}

\par\end{centering}

\textbf{\footnotesize a--b,}{\footnotesize{} more panels}
\end{figure*}


\newpage{}

\appendix
\pagestyle{plain}\setcounter{page}{1}%\newgeometry{}\setcounter{figure}{0}\renewcommand{\thefigure}{S\arabic{figure}}\setcounter{table}{0}\renewcommand{\thetable}{S\arabic{table}}


\section{Supplementary Information}


\subsection{Supplementary Figures}

\begin{figure}[H]
\begin{centering}
\includegraphics[height=1\textheight]{24_GD_dev_R_paperutils_inst_lyx_composite_fig.pdf}
\par\end{centering}

\begin{centering}
\caption{\label{fig:S_wtCartoons}Composite Figure}

\par\end{centering}

{\footnotesize Circuit models }
\end{figure}


\clearpage{}


\subsection{Supplementary Tables}

\begin{table}[h]
\begin{centering}
\caption{\textbf{\label{tab:S_cellcounts}}Some Table}

\par\end{centering}

\vspace{0.5cm}


\begin{centering}
\begin{tabular}{|c|c|c|c|}
\cline{3-4} 
\multicolumn{2}{c|}{} &
\textbf{Col 1} &
\textbf{Col 2}\tabularnewline
\hline 
\multirow{3}{*}{\textbf{\emph{Heading}}} &
Row 1 &
18.2 (1.7) &
11.2 (1.0)\tabularnewline
\cline{2-4} 
 & Row 2 &
12.8 (1.6) &
11.3 (0.9)\tabularnewline
\cline{2-4} 
 & Row 3 &
18.8 (1.5) &
11.7 (1.2)\tabularnewline
\hline 
\end{tabular}
\par\end{centering}

\vspace{0.5cm}


{\footnotesize Cell counts (mean (s.d.))}
\end{table}
\clearpage{}


\subsection{Supplementary Discussion}

Some more discussion.
\end{document}
